% LaTeX Curriculum Vitae Template
%
% Copyright (C) 2004-2009 Jason Blevins <jrblevin@sdf.lonestar.org>
% http://jblevins.org/projects/cv-template/
%
% You may use use this document as a template to create your own CV
% and you may redistribute the source code freely. No attribution is
% required in any resulting documents. I do ask that you please leave
% this notice and the above URL in the source code if you choose to
% redistribute this file.

\documentclass[letterpaper,uplatex]{article}

\usepackage{hyperref}
\usepackage{geometry}

% Comment the following lines to use the default Computer Modern font
% instead of the Palatino font provided by the mathpazo package.
% Remove the 'osf' bit if you don't like the old style figures.
\usepackage[T1]{fontenc}
\usepackage[sc,osf]{mathpazo}

% Set your name here
\def\name{木村 太一}

% Replace this with a link to your CV if you like, or set it empty
% (as in \def\footerlink{}) to remove the link in the footer:
\def\footerlink{http://jblevins.org/projects/cv-template/}

% The following metadata will show up in the PDF properties
\hypersetup{
  colorlinks = true,
  urlcolor = black,
  pdfauthor = {\name},
  pdfkeywords = {economics, statistics, mathematics},
  pdftitle = {\name: Curriculum Vitae},
  pdfsubject = {Curriculum Vitae},
  pdfpagemode = UseNone
}

\geometry{
  body={6.5in, 8.5in},
  left=1.0in,
  top=1.25in
}

% Customize page headers
\pagestyle{myheadings}
\markright{\name}
\thispagestyle{empty}

\usepackage{url}

% Custom section fonts
\usepackage{sectsty}
\sectionfont{\rmfamily\mdseries\Large}
\subsectionfont{\rmfamily\mdseries\itshape\large}

% Other possible font commands include:
% \ttfamily for teletype,
% \sffamily for sans serif,
% \bfseries for bold,
% \scshape for small caps,
% \normalsize, \large, \Large, \LARGE sizes.

% Don't indent paragraphs.
\setlength\parindent{0em}

% Make lists without bullets
\renewenvironment{itemize}{
  \begin{list}{}{
    \setlength{\leftmargin}{1.5em}
  }
}{
  \end{list}
}

\begin{document}

% Place name at left
{\LARGE \name {\normalsize (きむら たいち)}}

% Alternatively, print name centered and bold:
%\centerline{\huge \bf \name}

\vspace{0.25in}

\begin{minipage}{0.45\linewidth}
  \href{}{〒223-8526} \\
  神奈川県横浜市港北区日吉4-1-1 \\
  協生館研究室26
\end{minipage}
\begin{minipage}{0.45\linewidth}
  \begin{tabular}{ll}
    電話番号: & (045) 564-2478 \\
    内線: & 37522\\
    Email: & \href{mailto:tkimura@kbs.keio.ac.jp}{\tt tkimura@kbs.keio.ac.jp}\\
    Web site: & \url{https://ticmrk.github.io/}
  \end{tabular}
\end{minipage}

\section*{学歴}

\begin{itemize}
  \item 2016年3月 一橋大学大学院商学研究科博士後期課程修了(博士(商学))
  \item 2013年3月 一橋大学大学院商学研究科修士課程修了(修士(商学))
  \item 2011年3月 一橋大学商学部卒業(学士(商学))
\end{itemize}

\section*{職歴}

\begin{itemize}
     \item 2021年10月〜現在 慶應義塾大学大学院経営管理研究科専任講師
     \item 2018年4月〜2021年9月 慶應義塾大学大学院経営管理研究科専任講師(有期)
     \item 2016年4月〜2018年3月 慶應義塾大学大学院経営管理研究科助教(有期)
\end{itemize}

\section*{業績}

\subsection*{査読付き学術論文}

\begin{itemize}
    \item \underline{\textbf{Kimura, Taichi}} and Morimitsu, Takahiro. 2023. Cost--based Pricing in Government Procurement with Unobservable Cost--reducing Actions and Productivity. \textit{Asia--Pacific Journal of Accounting and Economics}, 30(2): 373-390.
    
    \item \underline{\textbf{木村 太一}}. 2019.業績評価情報の伝達・利用が組織アイデンティフィケーションに与える影響に関する定量的研究.『慶應経営論集』36(1): 39--56.

	\item 尻無濱 芳崇,\underline{\textbf{木村 太一}},劉 美玲,市原 勇一.2017.形成型尺度開発ガイドラインと管理会計研究への示唆.『一橋商学論叢』12(2): 62--71.

	\item \underline{\textbf{木村 太一}}.2014.組織文化概念を用いた管理会計研究の現状と展望『原価計算研究』30(2): 52-64.
\end{itemize}

\subsection*{査読無し学術論文}

\begin{itemize}
    \item \underline{\textbf{木村 太一}},村上 裕太郎.2022.「ピア・プレッシャーがインセンティブ設計に与える影響:分析的研究を中心とした文献レビュー」『産業経理』82(2): 111-124.

    \item 森光 高大,\underline{\textbf{木村 太一}}.2021.主観的業績評価におけるバイアス:分析的研究のレビューに基づく考察.『西南学院大学商学論集』68(1-2): 65--88.
    
	\item 劉 美玲,市原 勇一,\underline{\textbf{木村 太一}},尻無濱 芳崇.2015.管理会計研究における形成型尺度の利用と現状:構成概念の測定モデルの選択.『メルコ管理会計研究』8(1): 77-87.

	\item \underline{\textbf{木村 太一}}.2015.組織文化概念を援用した経験的な管理会計研究のレビュー : 組織文化観を軸として『企業会計』67(8): 1179-1185.
\end{itemize}

\subsection*{その他論文}

\begin{itemize}
    \item \underline{\textbf{Kimura, Taichi}} and Morimitsu, Takahiro. 2017. Government Procurement Contract Design for Encouraging Cost Reduction. \textit{Proceedings of 29th Asian--Pacific Conference on International Accounting Issue}.
\end{itemize}

\subsection*{書籍}

\begin{itemize}
    \item 山根 節,太田 康広,村上 裕太郎,\underline{\textbf{木村 太一}}.『ビジネス・アカウンティング<第5版>』中央経済社.
	\item \underline{\textbf{木村 太一}}.2021.非営利組織のキャリア・コンサーン.太田 康広(編著)『人事評価の会計学:キャリア・コンサーンと相対的業績評価』中央経済社,第4章,63-81頁.(分担執筆)
\end{itemize}

\subsection*{その他原稿}

\begin{itemize}
    \item \underline{\textbf{木村 太一}}.2023.「評価とバイアス,その解消」『三田評論』1282: 81.
\end{itemize}

\subsection*{学会発表}

\begin{itemize}
    \item 2023年10月: 32nd Asian-Pacific Conference on International Accounting Issues (Gold coast, Australia) ``On the use of calibration committees in subjective performance evaluation.''

    \item 2023年5月: 45th Annual Congress of European Accounting Association (Helsinki-Espoo, Finland) ``On the use of calibration committees in subjective performance evaluation.''
    
    \item 2022年8月: 第81回日本会計研究学会(東京大学,オンライン)「主観的業績評価におけるアピールの役割:分析的研究による考察」.
    
   \item 2022年8月: 2022年度日本管理会計学会(明治大学)「主観的業績評価におけるアピールとコストに関する分析」.

  \item 2020年9月: 第79回日本会計研究学会(北海道大学,オンライン)「非対称情報下における政府調達契約の設定」.
  
  \item 2019年10月: 31st Asian--Pacific Conference on International Accounting Issue (Warsaw, Poland) ``Conformity Pressure and Compensation Contracts.''

  \item 2019年9月: 第78回日本会計研究学会(神戸学院大学)「個別受注契約における目標原価の設定」.

  \item 2019年5月: 42nd Annual Congress of European Accounting Association (Paphos, Cyprus) ``Conformity Pressure and Compensation Contracts.''

  \item 2018年11月: 30th Asian--Pacific Conference on International Accounting Issue (San Francisco, USA) ``Diagreement and Performance Evaluation Systems.''

  \item 2018年9月: 第77回日本会計研究学会(神奈川大学)「組織成員の楽観性が業績評価システムに与える影響:数理モデル分析による検討」.

	\item 2017年11月: 29th Asian--Pacific Conference on International Accounting Issue (Kuala Lumpur, Malaysia) ``Government Procurement Contract Design for Encouraging Cost Reduction.''

	\item 2017年9月: 第76回日本会計研究学会(広島大学)「原価低減を促す防衛調達契約の設計」.

	\item 2017年2月: 日本原価計算研究学会2016年度関東部会(石巻専修大学)「非市場性物品の調達における最適な契約形態の設計」.

	\item 2015年9月: 第41回日本原価計算研究学会全国大会(日本大学)「業績評価指標のコミュニケーションの効果にかんする定量的研究:組織アイデンティフィケーションを媒介変数として」.

	\item 2013年8月: 第39回日本原価計算研究学会全国大会(専修大学)「管理会計研究における組織文化概念―文献レビューによる考察―」.
\end{itemize}

\subsection*{研究助成}

\begin{itemize}
\item 「目標達成を条件とした非線形な報酬契約の行動契約理論による分析」(21K13406)文部科学省: 科学研究費補助金(若手研究).研究期間:2021年4月-2024年3月.代表者:木村太一.

\item 「行動契約理論の観点からのマネジメント・コントロール理論の再検討」(18K12897)文部科学省: 科学研究費補助金(若手研究).研究期間:2018年4月-2021年3月.代表者:木村太一.

\item 「同質的な組織のマネジメント・コントロール:数理モデル分析による研究」(16H07173)文部科学省: 科学研究費補助金(研究活動スタート支援).研究期間:2016年8月-2018年3月.代表者:木村太一.

\item 「マネジャーが有する組織への一体感(identification)に対して業績測定システムが与える影響にかんする実証研究」メルコ学術振興財団:2013年度研究助成金.研究期間:2014年1月-12月.代表者:木村太一.
\end{itemize}

\subsection*{査読経験}
    \begin{itemize}
        \item 原価計算研究,現代ディスクロージャー研究
    \end{itemize}

\section*{教育実績}

\subsection*{担当授業}

\begin{itemize}
    \item 会計管理(慶應義塾大学大学院経営管理研究科,MBA基礎):2016年〜現在.
    \item 会計管理(慶應義塾大学大学院経営管理研究科,EMBAコア):2018年〜現在.
    \item 経営管理会計(慶應義塾大学大学院経営管理研究科,MBA専門):2018年〜現在.
    \item マネジメント・コントロール特殊講義(慶應義塾大学大学院経営管理研究科,MBA):2018年〜現在.
    \item マネジメント・コントロール演習(慶應義塾大学大学院経営管理研究科,MBA):2018年〜現在.
    \item 個人研究A,B,C(慶應義塾大学大学院経営管理研究科,EMBA):2018年〜現在.
\end{itemize}

\subsection*{作成ケース教材}

\begin{itemize}
    \item 「管理会計計算ノート」2021.
    \item 「綱島金属株式会社」2021(標準原価計算).
    \item 「小机タイヤ株式会社」2021(ABC).
   \item 「原価計算の基礎知識」2020. 
   \item 「三菱電機」2018(森光高大と共著,コスト・シフティング).
   \item 「JALの再建とアメーバ経営」2016(アメーバ経営).
\end{itemize}

\bigskip

% Footer
\begin{center}
  \begin{footnotesize}
    Last updated: \today \\
    %qqqqqqqqqq\href{\footerlink}{\texttt{\footerlink}}
  \end{footnotesize}
\end{center}

\end{document}
